\newpage

\paragraph{\LARGE Aufgabe 6 - Multi-Threading und Synchronisation}

\section{Aufgabenstellung}
	\begin{quote}
		\begin{enumerate}
			\item Dem Programm wird per Kommandozeile ein Ordnerpfad \"ubergeben.\\
			\item Arbeiten Sie mit mehreren Threads. Ein Thread (Leser-Thread) liest die Dateien in dem \"ubergebenen Ordner ein und h\"angt den Inhalt sowie den Dateipfad an eine Queue an. Nutzen Sie daf\"ur eine struct Job.\\ 
			\item Eine zur Compileteit konfigurierbare Anzahl von Threads (Komprimierungs-Thread) liest jeweils einen Job aus der Queue, komprimiert seinen Inhalt und speichert diesen im Format <alter Dateiname>.compr. Dies soll solange wiederholt werden, bis die Queue leer und der Leser-Thread beendet ist.\\
			\item Der Leser-Thread soll Dateien, die mit .compr enden, ignorieren.\\
			\item Ein Komprimierungs-Thread bekommt bei seiner Erstellung als Parameter eine Instanznummer zugeordnet, die ihn z.B. bei Debug-Ausgaben eindeutig identifiziert.\\
			\item Die Zugriffe auf die Queue m\"ussen synchronisiert, d.h. gegeneinander gesch\"utzt sein.\\
			\item F\"ugen Sie im Leser-Thread nach dem Einlesen einer Datei ein sleep(1) und in den Kompressions-Threads nach den Komprimieren ein sleep(3) ein, um eine langsame Festplatte und einen komplexen Kompressionsalgorithmus zu simulieren. Beobachten Sie, wie ihr Programm mit und ohne die sleep-Anweisungen arbeitet.\\
			\item Bestimmen Sie n\"aherungsweise die Laufzeit bei unterschiedlicher Anzahl von Kompressions-Threads, z. B. mit der Funktion difftime() in main()\\
		\end{enumerate}
	\end{quote}
\newpage
\section{Aufgabe 6.1}
	\subsection{Vorbereitung}
		\begin{quote}
			C-Projekt anlegen.\\
			Makefile schreiben.\\
			Include-Dateien includieren.\\ 
		\end{quote}
	\subsection{Durchführung}
		\begin{quote}
			Code schreiben und dann testen bzw debuggen.\\
		\end{quote}
	\subsection{Fazit}
		\begin{quote}
			\small\lstinputlisting[language=C, firstline=34, lastline=45]{../src/main.c}
			Der Ordnerpfad ist das zweite argument das der Main-Methode \"ubergeben wird. Das Programm wird mit dem Befehl ./main <Verzeichnis> gestartet\\
		\end{quote}

\section{Aufgabe 6.2}
	\subsection{Vorbereitung}
		\begin{quote}
			\lstinputlisting[language=C, firstline=21, lastline=29]{../src/main.c}
			struct Job und Joblist schreiben.\\
		\end{quote}
	\subsection{Durchführung}
		\begin{quote}
			Code schreiben und dann testen bzw debuggen.\\
		\end{quote}
	\subsection{Fazit}
		\begin{quote}
			\tiny\lstinputlisting[language=C, firstline=63, lastline=112]{../src/main.c}
			Zuerst wird die Thread-Methode f\"ur den Leser erstellt. Mit opendir wird das \"ubergebene Verzeichnis ge\"offnet. Mit readdir wird das Verzeichnis gelesen. Mit ''char *filename = (*dirptr).d\_name;'' wird der Name der Datei ermittelt. Die wird dann mit fopen ge\"offnet und danach mit getline Zeilenweise eingelesen. Die ermittelten Daten werden in der struct Job gespeichert und dann mit ''queue\_insert(joblist.jobs, job);'' der Queue hizugef\"ugt. Dann werden mit pthread\_create in der main Methode die Threads erstellt und mit pthread\_join auf sie gewartet.\\
		\end{quote}

\section{Aufgabe 6.3}
	\subsection{Vorbereitung}
		\begin{quote}
			keine
		\end{quote}
	\subsection{Durchführung}
		\begin{quote}
			Code schreiben und dann testen bzw debuggen.
		\end{quote}
	\subsection{Fazit}
		\begin{quote}
			\tiny\lstinputlisting[language=C, firstline=34, lastline=42]{../src/main.c}
			Mit hilfe eines dritten Kommandozeilenparameters kann die Anzahl der Kommrimier-Threads zur Compilezeit festgelegt werden. Wenn der dritte Kommandozeilenparameter fehlt wird ein Standardwert festgelegt. 
			\tiny\lstinputlisting[language=C, firstline=114, lastline=152]{../src/main.c}
			Zuerst wird eine Thread-Methode f\"urs Komprimieren erstellt. Mit opendir wird das \"ubergebene Verzeichnis ge\"offnet. Dann wird mit ''job = queue\_head(joblist.jobs);'' die erste Datei aus der Queue gelolt und danach aus der Queue gel\"oscht. Dann wird der Inhalt vom job mit compress\_string Komprimiert. Dann wird die Datei dessen Inhalt grade Bearbeitet wurde mit rename in <alter Name>.compr umbenannt. Danach wird der Kommprimierte String mit fprintf in die Umbenannte Datei geschrieben. Mit ''while(joblist.jobs->head != 0 || statuslt != 0)'' (statuslt ist der Status des Lese-Threads) wird sicher gestllt dass der Thread sollange arbeitet bis die Queue leer ist und der Lese-Thread beendet ist.\\
		\end{quote}

\section{Aufgabe 6.4}
	\subsection{Vorbereitung}
		\begin{quote}
			keine\\
		\end{quote}
	\subsection{Durchführung}
		\begin{quote}
			Code schreiben und dann testen bzw debuggen.\\
		\end{quote}
	\subsection{Fazit}
	\begin{quote}
			\tiny\lstinputlisting[language=C, firstline=79, lastline=81]{../src/main.c}
			Mit der Methode strrchr wird die Dateiendung ermittelt. Die wird dann mit strcmp auf die endung ''.compr'' verglichen wenn der Vergleich true ergibt dann wird die Datei \"ubersprungen.\\
	\end{quote}

\section{Aufgabe 6.5}
\subsection{Vorbereitung}
\begin{quote}
	keine\\
\end{quote}
\subsection{Durchführung}
\begin{quote}
	Code schreiben und dann testen bzw debuggen.\\
\end{quote}
\subsection{Fazit}
\begin{quote}
	Die Komprimierungs-Threads sind in einen Array gespeichert und haben einen eindeutigen Index.\\
\end{quote}

\section{Aufgabe 6.6}
\subsection{Vorbereitung}
\begin{quote}
	keine\\
\end{quote}
\subsection{Durchführung}
\begin{quote}
	Code schreiben und dann testen bzw debuggen.\\
\end{quote}
\subsection{Fazit}
\begin{quote}
	Die Queue ist mit einen Mutex gesch\"utzt der mit ''pthread\_mutex\_init(joblist.mutex, NULL);'' initialisiert wird. Mit ''pthread\_mutex\_lock(joblist.mutex);'' in den Thread-Methoden die Queue gesperrt wird und mit pthread\_mutex\_unlock wider entsperrt wird. Und schliesslich mit ''pthread\_mutex\_destroy(joblist.mutex);'' zerst\"ort wird.\\
\end{quote}

\section{Aufgabe 6.7}
\subsection{Vorbereitung}
\begin{quote}
	keine
\end{quote}
\subsection{Durchführung}
\begin{quote}
	Code schreiben und dann testen bzw debuggen.
\end{quote}
\subsection{Fazit}
\begin{quote}
	
\end{quote}


\section{Aufgabe 6.8}
\subsection{Vorbereitung}
\begin{quote}
	keine
\end{quote}
\subsection{Durchführung}
\begin{quote}
	Code schreiben und dann testen bzw debuggen.
\end{quote}
\subsection{Fazit}
\begin{quote}
	
\end{quote}