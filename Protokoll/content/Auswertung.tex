\newpage

\paragraph{\LARGE Aufgabe 6 - Multi-Threading und Synchronisation}

\section{Aufgabenstellung}
	\begin{quote}
		\begin{enumerate}
			\item Dem Programm wird per Kommandozeile ein Ordnerpfad \"ubergeben.\\
			\item Arbeiten Sie mit mehreren Threads. Ein Thread (Leser-Thread) liest die Dateien in dem \"ubergebenen Ordner ein und h\"angt den Inhalt sowie den Dateipfad an eine Queue an. Nutzen Sie daf\"ur eine struct Job.\\ 
			\item Eine zur Compileteit konfigurierbare Anzahl von Threads (Komprimierungs-Thread) liest jeweils einen Job aus der Queue, komprimiert seinen Inhalt und speichert diesen im Format <alter Dateiname>.compr. Dies soll solange wiederholt werden, bis die Queue leer und der Leser-Thread beendet ist.\\
			\item Der Leser-Thread soll Dateien, die mit .compr enden, ignorieren.\\
			\item Ein Komprimierungs-Thread bekommt bei seiner Erstellung als Parameter eine Instanznummer zugeordnet, die ihn z.B. bei Debug-Ausgaben eindeutig identifiziert.\\
			\item Die Zugriffe auf die Queue m\"ussen synchronisiert, d.h. gegeneinander gesch\"utzt sein.\\
			\item F\"ugen Sie im Leser-Thread nach dem Einlesen einer Datei ein sleep(1) und in den Kompressions-Threads nach den Komprimieren ein sleep(3) ein, um eine langsame Festplatte und einen komplexen Kompressionsalgorithmus zu simulieren. Beobachten Sie, wie ihr Programm mit und ohne die sleep-Anweisungen arbeitet.\\
			\item Bestimmen Sie n\"aherungsweise die Laufzeit bei unterschiedlicher Anzahl von Kompressions-Threads, z. B. mit der Funktion difftime() in main()\\
		\end{enumerate}
	\end{quote}
\newpage
\section{Aufgabe 6.1}
	\subsection{Vorbereitung}
		\begin{quote}
			C-Projekt anlegen.
			Makefile schreiben.
		\end{quote}
	\subsection{Durchführung}
		\begin{quote}
			Code schreiben und dann testen bzw debuggen.
		\end{quote}
	\subsection{Fazit}
		\begin{quote}
			
		\end{quote}

\section{Aufgabe 6.2}
	\subsection{Vorbereitung}
		\begin{quote}
			keine
		\end{quote}
	\subsection{Durchführung}
		\begin{quote}
			Code schreiben und dann testen bzw debuggen.
		\end{quote}
	\subsection{Fazit}
		\begin{quote}
			
		\end{quote}

\section{Aufgabe 6.3}
	\subsection{Vorbereitung}
		\begin{quote}
			keine
		\end{quote}
	\subsection{Durchführung}
		\begin{quote}
			Code schreiben und dann testen bzw debuggen.
		\end{quote}
	\subsection{Fazit}
		\begin{quote}
			
		\end{quote}

\section{Aufgabe 6.4}
\subsection{Vorbereitung}
\begin{quote}
	keine
\end{quote}
\subsection{Durchführung}
\begin{quote}
	Code schreiben und dann testen bzw debuggen.
\end{quote}
\subsection{Fazit}
\begin{quote}
	
\end{quote}

\section{Aufgabe 6.5}
\subsection{Vorbereitung}
\begin{quote}
	keine
\end{quote}
\subsection{Durchführung}
\begin{quote}
	Code schreiben und dann testen bzw debuggen.
\end{quote}
\subsection{Fazit}
\begin{quote}
	
\end{quote}

\section{Aufgabe 6.6}
\subsection{Vorbereitung}
\begin{quote}
	keine
\end{quote}
\subsection{Durchführung}
\begin{quote}
	Code schreiben und dann testen bzw debuggen.
\end{quote}
\subsection{Fazit}
\begin{quote}
	
\end{quote}

\section{Aufgabe 6.7}
\subsection{Vorbereitung}
\begin{quote}
	keine
\end{quote}
\subsection{Durchführung}
\begin{quote}
	Code schreiben und dann testen bzw debuggen.
\end{quote}
\subsection{Fazit}
\begin{quote}
	
\end{quote}


\section{Aufgabe 6.8}
\subsection{Vorbereitung}
\begin{quote}
	keine
\end{quote}
\subsection{Durchführung}
\begin{quote}
	Code schreiben und dann testen bzw debuggen.
\end{quote}
\subsection{Fazit}
\begin{quote}
	
\end{quote}