\newpage

\paragraph{\LARGE Aufgabe 10 - Kernel Treiber}

\section{Aufgabenstellung}
	\begin{quote}
		Kompilieren Sie das Modul lkm.c und machen Sie sich mit den Befehlen zum Laden bzw. Entladen von Modulen vertraut.\\ \\
		\begin{enumerate}
			\item Vollziehen Sie den Aufbau eines Kernel Moduls nach und erg\"anzen Sie das im Info-Blatt vorgestellte Modul memory.c:\\
			\item \"Andern Sie die Major Nummer.\\
			\item Erweitern Sie das Kernel Modul so, dass 256 Byte im Buffer gespeichert werden k\"onnen (antatt 1 Byte).\\
			\item Erweitern Sie das Kernel Modul so, dass der Speicherinhalt des Buffers byteweise ausgelesen werden kann.\\
			\item Schreiben Sie ein Programm, dass sowohl die Schreib- als auch die Lesemethode des Kerneltreibers verwendet.\\
			\item Der zu speichernde Text soll beim Schreiben in den Kernel Space vom Kernel Modul mit Hilfe des printk-Befehls im Syslog-D\"amon ausgegeben werden.\\
			\item Unter welchen Bedingungen kann der vorgestellt Treiber verwendet werden, um auf reale Hardware zuzugreifen. Welche Anpassungen sind dazu notwendig?\\
		\end{enumerate}
	\end{quote}
\newpage
\section{Aufgabe 10.1}
	\subsection{Vorbereitung}
		\begin{quote}
			C-Projekt anlegen.
			Makefile schreiben.
		\end{quote}
	\subsection{Durchführung}
		\begin{quote}
			Code schreiben und dann testen bzw debuggen.
		\end{quote}
	\subsection{Fazit}
		\begin{quote}
			
		\end{quote}