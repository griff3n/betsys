\newpage

\paragraph{\LARGE Aufgabe 3 - Papierkorb unter Unix}
\section{Aufgabenstellung allgemein}
	\begin{quote}
		Der Papierkorb soll die folgenden Anforderungen erf\"ullen:
		\begin{enumerate}
			\item Der Papierkorb ist der Ordner \$HOME/.trashBin\\
			\item Jede durch den Benutzer ''gel\"oschte'' Datei wird in den Papierkorb\\
			geschoben, wobei der Name der Datei durch einen eindeutigen Namen\\ der Form 080321195131\_18333.dat ersetzt wird. Die Ziffernfolge am\\
			Anfang ist ein Zeitstempel, der durch das Kommando date '+\%y\%m\%d\%H\%M\%S'\\
			erzeugt werden kann. Die Ziffernfolge nach \_ ist die Prozess-ID, 	die als\\
			Wert in der Shellvariablen \$\$ enthalten ist.\\
			\item Zur Verwaltung der ''gel\"oschten'' Dateien wird die Verzeichnisdatei\\
			\$HOME/.trashBin/.dir benutzt. F\"ur jede ''gel\"oschte'' Datei enth"alt sie eine Zeile der Form\\ \\
			080321195131\_18333.dat! /home/someuser/BS/somefile\\ \\
			Es ist also der Dateiname im Papierkorb und der urspr\"ungliche volle Pfadname der Datei eingetragen.\\
		\end{enumerate}
	\end{quote}

\section{Aufgabe 3.1}
	\subsection{Aufgabenstellung}
		\begin{quote}
			Ein Shellskript delete zum ''L\"oschen''. Dabei wird eine Datei, die als Argument\\
			\"ubergeben wird, in das Papierkorbverzeichnis verschoben, wobei wie oben\\
			beschrieben ein neuer (eindeutiger) Name gebildet wird. Die Verzeichnisdatei\\
			.dir muss nat\"urlich um eine entsprechende Zeile erweitert werden.\\ \\
		\end{quote}
	\subsection{Vorbereitung}
		\begin{quote}
			Ordner .trashBin erstellen\\ \\
		\end{quote}
	\subsection{Durchführung}
		\begin{quote}
			Skript schreiben und dann testen.\\ \\
		\end{quote}
	\subsection{Fazit}
		\begin{quote}
			Zuerst wird mit ''if {[} ! -d ''\$HOME/.trashBin/'' {]}'' gepr\"uft ob der Papierkorb existiert und wenn n\"otig mit mkdir erstellt.\\
			Und dann wird mit ''if {[} -e ''\$file'' {]}'' gepr\"uft ob die \"ubergebene Datei existiert.\\
			Dann wird mit ''date=\$(date '+\%y\%m\%d\%H\%M\%S')''\\
			ein Zeitstempel erzeugt und in die Variable date geschrieben und die Prozess-ID\\
			wird in der Variable prozessid gespeichert. Mit dem Befehl\\
			''mv \$file ''\$HOME/.trashBin/\$date''''\_\$prozessid.dat''''\\
			wird die Datei ''gel\"oscht''.\\
			Danach wird mit	''echo ''\$date''''\_\$prozessid.dat! \$file'' >> \$HOME/.trashBin/.dir''\\
			ein Eintrag in die Verzeichnisdatei .dir erzeugt.\\ \\
		\end{quote}

\section{Aufgabe 3.2}
	\subsection{Aufgabenstellung}
		\begin{quote}
			Sehen Sie eine Ausgabe vor, die den Namen der tempor\"aren Datei im Papierkorb ausgibt.\\ \\
		\end{quote}
	\subsection{Vorbereitung}
		\begin{quote}
			keine\\ \\
		\end{quote}
	\subsection{Durchführung}
		\begin{quote}
			Skript schreiben und dann testen.\\ \\
		\end{quote}
	\subsection{Fazit}
		\begin{quote}
			Mit ''echo ''\$date''''\_\$prozessid.dat'''' wird der Name der tempor\"aren Datei im\\
			Papierkorb ausgegeben.\\ \\
		\end{quote}
\newpage
\section{Aufgabe 3.3}
	\subsection{Aufgabenstellung}
		\begin{quote}
			Ein Shellskript undelete zum Wiederherstellen einer ''gel\"oschten'' Datei.\\
			Dem	Skript wird als Argument der Name der Datei im Papierkorb (ohne\\
			Pfadnamen!) \"ubergeben (also z. B. undelete 070321195131\_18333.dat).\\
			Die Datei wird dann unter ihrem urspr\"unglichen Pfad wiederhergestellt,\\
			die Papierkorbdatei gel\"oscht und die entsprechnde Zeile in der Verzeichnisdatei\\
			entfrent. Existiert der urspr\"ungliche Pfad nicht mehr, so soll eine\\
			Fehlermeldung ausgegeben werden.\\ \\
		\end{quote}
	\subsection{Vorbereitung}
		\begin{quote}
			keine\\ \\
		\end{quote}
	\subsection{Durchführung}
		\begin{quote}
			Skript schreiben und dann testen.\\ \\
		\end{quote}
	\subsection{Fazit}
		\begin{quote}
			Zuerst wird \"uberpr\"uft ob der Papierkorb existiert. Dann wird getestet ob die \"ubergebene Datei im Papierkorb ist.
			Wenn ja  dann wird mit ''zeilen=\$(wc -l ''\$HOME/.trashBin/.dir'' | cut -d'' '' -f1)'' die Anzahl der zeilen ermittelt.
			Danach wird Zeile f\"ur Zeile nach der Datei gesucht.
			Wenn sie gefunden wurde dann wird gepr\"uft ob das Quellverzeichnichs existiert. Wenn es existiert wird mit mv die gel\"oschte Datei wieder hergestellt ansonsten wird eine Fehlermeldung angezeigt.
			Wenn die Linie nicht gleich der Datei ist oder das Quellverzeichnichs nicht existiert wird mit ''\$tdir = \$tdir + \$line + ''\textbackslash n\textbackslash c'' die Zeile zwischen gespeichert um am Ende mit
			''echo \$tdir > ''\$HOME/.trashBin/.dir'''' die .dir zu speichern.
			
		\end{quote}

