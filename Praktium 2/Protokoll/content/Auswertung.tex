\newpage

\paragraph{\LARGE Aufgabe 2 - Flusskontrolle in der Bash}

\section{Aufgabe 2.1}
	\subsection{Aufgabenstellung}
		\begin{quote}
			Schreiben Sie ein Shell-Skript das mit Hilfe von Alternativen folgende Informationen zu\\
			einen Pfadnamen (relatix oder absolut), der als Argument \"ubergeben wird, ausgibt:\\ \\
			\begin{itemize}
				\item die Datei oder das Verzeichnis existiert oder nicht;\\
				\item es handelt sich um eine regul\"are Datei oder ein Verzeichnis;\\
				\item es handelt sich um einen symbolischen Link;\\
				\item der Aufrufer ist der Besitzer der Datei;\\
				\item den Besitzer der Datei oder des Verzeichnisses.\\ \\
			\end{itemize}
			Die einzelnen \"Uberpr\"ufungen sollen dabei jeweils als Funktionen realisiert werden.\\
			Verwenden Sie f\"ur die Tests die Ausdr\"ucke wie sie im Kapitel 7.1 des Bash Beginners Guide\\
			angegeben sind. (Bemerkung: Sie k\"onnen z.B. den cut Befehl, den awk Befehl, den sed\\
			Befehl oder . . . verwenden)\\ \\
		\end{quote}
	\subsection{Vorbereitung}
		\begin{quote}
			Kapitel 7.1 des Bash Beginners Guide lesen und sich \"uber shellskripting informieren.\\ \\
		\end{quote}
	\subsection{Durchführung}
		\begin{quote}
			 Skript schreiben und dann testen.\\ \\
		\end{quote}
	\subsection{Fazit}
		\begin{quote}
			Mit dem Operator -e des Test-Kommandos kann man pr\"ufen ob die Datei existiert.\\
			Mit dem Operator -d des Test-Kommandos kann man pr\"ufen ob die Datei ein Verzeichnis ist.\\
			Mit dem Operator -f des Test-Kommandos kann man pr\"ufen ob die Datei eine regul\"are Datei ist.\\
			Mit dem Operator -h des Test-Kommandos kann man pr\"ufen ob die Datei ein symbolischer Link ist.\\
			Mit dem Operator -O des Test-Kommandos kann man pr\"ufen ob der Aufrufer der Besitzer der Datei ist.\\
			Mit dem Befehl: \\ \\
			\begin{quote}
				ls -lad FILE | cat -d'' '' -f4\\ \\
			\end{quote}
			wird der Besitzer der Datei ausgegeben.\\ \\
		\end{quote}
	\subsection{Quellen}
		\begin{quote}
			\begin{itemize}
				\item http://tldp.org/LDP/Bash-Beginners-Guide/html/sect\_07\_01.html Zugriff: 19.04.2016\\
				\item https://wiki.ubuntuusers.de/Shell/Bash-Skripting-Guide\_f\%C3\%BCr\_Anf\%C3\%A4nger/ Zugriff: 19.04.2016\\
			\end{itemize}
		\end{quote}
\newpage
\section{Aufgabe 2.2}
	\subsection{Aufgabenstellung}
		\begin{quote}
			Erweitern Sie dieses Shell-Skript, so dass der Aufrufer eine beliebige Anzahl von\\
			Pfadnamen \"ubergeben kann, die in einer Schleife (for oder while) abgearbeitet werden.\\
			Verwenden Sie die Shell-Variable \$@ f\"ur die Liste der Argumente des Skripts.\\
			Testen Sie Ihr Skript mit verschiedenen Eingaben.\\ \\
		\end{quote}
	\subsection{Vorbereitung}
		\begin{quote}
			Aufgabe 1 l\"osen.\\ \\
		\end{quote}
	\subsection{Durchführung}
		\begin{quote}
			Skript schreiben und dann testen.\\ \\
		\end{quote}
	\subsection{Fazit}
		\begin{quote}
			Wenn die Liste bei einer for Schleife weggelssen wird dann wird automatisch die Shell-Variable \$@ verwendet.\\
			Es muss nur noch bei jeden Schleifendurchlauf das Skript aus Aufgabe 1 mit ein Parameter aufgerufen werden.\\
		\end{quote}
\newpage
\section{Aufgabe 2.3}
	\subsection{Aufgabenstellung}
		\begin{quote}
			Erweitern Sie das Skript aus 2. so, dass nach der Ausgabe der Informationen zu der Datei\\
			eine Pr\"ufung der Dateiendung erfolgt. Endet die Datei auf .txt, so wird der Nutzer in\\
			einem Dialog gefragt, ob er die Datei angezeigt haben m\"ochte oder nicht. Im positiven\\
			Fall wird die Datei mit den Ihnen bekannten UNIX-Kommandos auf der Shell ausgegeben.\\ \\
		\end{quote}
	\subsection{Vorbereitung}
		\begin{quote}
			Aufgabe 1 und Aufgabe 2 l\"osen.\\ \\
		\end{quote}
	\subsection{Durchführung}
		\begin{quote}
			Skript schreiben und dann testen.\\ \\
		\end{quote}
	\subsection{Fazit}
		\begin{quote}
			Mit dieser Zeile:\\ \\
			\begin{quote}
				txt=\$\{file\#\#*.\}\\ \\
			\end{quote}
			wird bezweckt dass alles was links vom Punkt ist weg geschnitten wird.\\
			Danach wird gepr\"uft op der daraus resultierende String \"aquivalent mit den String txt ist.\\  
			Wenn der String gleich ist wird mit den Befehl:\\ \\
			\begin{quote}
				d -p ''\$file ist ein Textdokument soll sie angezeigt werden.\\ Ja/Nein'' answer\\ \\
			\end{quote}
			gefragt ob die Datei angezeigt werden soll.\\
			Wenn der Benutzer mit Ja antwortet dann wird die Datei mit less angezeigt\\
		\end{quote}